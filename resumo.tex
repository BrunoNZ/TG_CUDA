\chapter*{Resumo}

Neste trabalho farei um estudo de caso do processo de conversão de um algoritmo de processamento de dados da linguagem C para CUDA, com o intuito de mostrar que tal tarefa traz um grande ganho de desempenho sem exigir um conhecimento prévio em programação paralela por parte do programador. Tal objetivo foi pensado especialmente para que os resultados do trabalho possam servir como uma introdução a programação paralela baseada na arquitetura CUDA para qualquer pessoa envolvida em estudos na área atmosférica, desde um pesquisador mais experiente até o mais novo aluno de iniciação científica. O trabalho tem como conclusão um modelo de código genérico que pode ser usado como base para a conversão de novos algoritmos, e demonstra durante o desenvolvimento os problemas e soluções encontrados durante o processo de escrita desse código, além de detalhar as principais características dos algoritmos e dados de entrada usados como base para estudos atmosféricos.