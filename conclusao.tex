\chapter*{Conclusão}

Podemos concluir com esse trabalho que com apenas algumas modificações, necessárias para incluir as premissas básicas para que o programa execute em GPUs NVidia capacitadas com a tecnologia CUDA, podemos reduzir drasticamente o tempo de execução de algoritmos de processamento e análise de dados atmosféricos sem abrir mão da confiabilidade dos resultados. Além disso, o trabalho demonstra que tal método de conversão possui um grande potencial de se adaptar a outros algoritmos que possuem características parecidas com o demonstrado, o que facilita ainda mais a conversão destes.

Como principal contribuição para a comunidade acadêmica, principalmente aos pesquisadores da área atmosférica, o trabalho traz um modelo de código genérico completo que resolve o problema proposto e que pode ser usado como base para a conversão de outros algoritmos, além de explicar passo-a-passo o código apresentado, fazendo com que seja possível até mesmo pessoas com pouquíssima experiência em programação a se aventurarem no mundo da programação paralela.