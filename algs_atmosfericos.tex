\chapter{Algoritmos atmosféricos}

O objetivo desses algoritmos é processar uma base de dados, com base em conceitos estatísticos, de tal forma a gerar novas informações que possam ser usadas em pesquisas e estudos relacionada a área atmosférica. Grande parte desses algoritmos se baseiam no processamento de uma única série de dados e calculam a informação para apenas essa série, da mesma forma como é feito quando calcula-se a média de uma série de valores. Na verdade, muito desses algoritmos também podem serem usados em outras áreas com a diferença do tipo do dado de entrada usado e alguns pequenos ajustes. Posteriormente esses algoritmos são aplicados para um conjunto de séries relacionadas a diferentes locais da área a ser estudada, como será descrito nos próximos capítulos. Um fato importante a se notar é de que, normalmente, o processamento de cada uma das séries é independente uma das outras. 

Para exemplificar usaremos os algoritmos Filtro de Lanczos e Teste de Monte Carlo, pois cada um possui caraterísticas próprias que fazem com que a execução dos mesmos possa ser muito demorada quando executados sequencialmente.

\section{Filtro de Lanczos}

O algoritmo Filtro de Lanczos, descrito por [Duchon 1979], é utilizado, entre outras coisas, para filtrar variações temporais de dados atmosféricos diários. O problema consiste em calcular resultados para um certo vetor de dados, multiplicando-se um vetor de constantes de tamanho menor, $WT$, previamente calculado, com um pedaço da serie de dados, e armazenando o resultado numa nova série de dados, na posição central, como descrito na fórmula a seguir:

\[ Res[T+ \lfloor \frac{K}{2} \rfloor ]=\sum _{i=0} ^{k-1} WT[i] \times Dados[T+i] \]

Para cada resultado produzido por esse algoritmo é necessário $K$ multiplicações, onde $K$ é o número de pesos usados, definidos [Duchon 1979]. Logo, para um série de 100 valores é necessário aproximadamente $K*100$ operações. Uma vez que os dados de entrada usados para esses algoritmos possuem centenas de séries de centenas de valores cada uma, como será descrito mais a frente, o tempo de execução desse algoritmo, mesmo sendo um sequencial um função relativamente simples, tende a ser muito grande.

\section{Teste de Monte Carlo}

O teste de Monte Carlo é utilizado, dentro da área de estudos atmosféricos, para calcular a significância entre duas séries distintas. Esse algoritmo faz parte de uma classe de algoritmos que utilizam o método de Monte Carlo, o qual se baseia na observação de valores aleatórios e o uso dessa amostra para o cálculo da função de interesse. Outra aplicação muito comum desse método é o cálculo de integrais.

O algoritmo a ser estudado nesse trabalho é explicada, de acordo com o João Paulo Saboia, pelos seguintes passos, sendo $n_{e}$ o número de experimentos do teste de Monte Carlo:

\begin{enumerate}
\item  é calculado o coeficiente de correlação entre as séries A e B. Por simplicidade, a correlação entre essas duas séries será chamada de correlação original;

\item  a séria A sofre permutação entre seus membros, a fim de se formar uma nova série;

\item  é calculado o coeficiente de correlação entre esta nova série e a série B;

\item  compara-se o novo coeficiente de correlação com o anterior;

\item  repete-se os passos 2,3 e 4, $n_{e}$ vezes;

\item  chamando de $cor_{m}$ o número de vezes em que o novo coeficiente de correlação foi maior do que o original, o nível de significância é definido como a razão entre $cor_{m}$ e $n_{e}$;
\end{enumerate}

A execução desse algoritmo pode ser bastante elevado dependendo do valor $n_{e}$, pois o mesmo define a quantidade de novas séries a serem geradas e o cálculo de um novos coeficiente de correlação. Sabendo-se que essas séries podem conter centenas de valores, a quantidade de acessos a memória necessário para escrever e ler todos esses valores é muito grande, elevando o tempo de execução consideravelmente.