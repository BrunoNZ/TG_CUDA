\chapter{Proposta de um esqueleto de código padrão}

Por fim, resumindo-se tudo o que foi já foi dito e usando como base a implementação em CUDA, que se mostrou a mais eficiente, é apresentado um esqueleto padrão de código que soluciona o problema proposto de forma simplificada. Esse esqueleto visa facilitar o trabalho de adaptação de algoritmos sequenciais de análise e processamento de dados atmosféricos em uma solução paralela do mesmo. 

A implementação contém duas estruturas de variáveis, definidas no arquivo \ref{lst:codigo_h}, nomeadas \texttt{parametros} e \texttt{parametros\_exec} que armazenam, respectivamente, os parâmetros de entrada e os parâmetros de execução da função em CUDA. Por sua vez, os parâmetros de execução são calculados pela função \texttt{calcula\_parametros\_execucao}, que tem como argumentos o número de pontos e o tamanho das séries do dado de entrada. Os parâmetros de entrada são lidos pela função \texttt{le\_parametros\_entrada}.

Os dados de entrada são lidos e reorganizados, da forma descrita no capitulo \ref{cap:proposta_solucao}, pela função \texttt{le\_matriz\_entrada} e são apontados pela variável \texttt{d\_entrada}. Após a leitura dos dados, inicia-se o processo de inicialização e configuração do device para que o mesmo seja capaz de executar as tarefas necessárias. Esse processo começa com a alocação de espaço na memória do device para armazenar os dados de entrada e saída, igualmente como é feito em qualquer outro programa, e para isso é utilizado a função \texttt{cudaMalloc} da biblioteca CUDA, a qual se assemelha muito com a função \texttt{malloc}. A grande diferença desse passo para o que normalmente é feito é que o espaço alocado não é para armazenar todo o dado de entrada mas apenas uma parte desse referente ao que será processado em cada ciclo.

Após esse passo inicia-se o laço de ciclos. Em cada um desses ciclos serão executados os passos descritos na seção \ref{cap:implementacao_solucao}. Para copiar os dados entre o host e o device é usado a função \texttt{cudaMemcpy}. Para executar o algoritmo em si é necessário passar como parâmetro, além dos parâmetros usuais, a quantidade de blocos por grid e de threads por bloco. Essas informações serão usadas pelo device para organizar as threads a serem executadas dentro hardware disponível no device. O número de threads por bloco é definida no início do arquivo \ref{lst:codigo_h} como uma constante de nome \texttt{THREADS\_POR\_BLOCO}. A alteração desse valor pode trazer uma melhora de desempenho, no entanto depende diretamente do device em que o código será executado. Para facilitar o uso do código foi escolhido um valor padrão com o qual foi obtido em média os melhores resultados em todos os devices testados. Já o número de threads por bloco é calculada pela função \texttt{calcula\_parametros\_execucao} levando em conta o número total de blocos e a quantidade de pontos a serem processados a cada ciclo. Por fim, a função \texttt{cudaDeviceSynchronize} garante que a execução do código principal só continue quando o todo o processamento no device termine para evitar que resultados ainda não calculados sejam copiados do device para o host.

Quando todos os ciclos terminarem, os resultados são escritos em disco pela função \linebreak \texttt{salva\_arq\_saida} no mesmo formato do dado de entrada original e os espaços em memória são liberados.

%%%%%%%%%%%%%%%%%%%%%%%%%%%%%%%%%%%%%%%%%%%%%
%Definicao do estilo dos codigos
%Baseado em:
%http://en.wikibooks.org/wiki/LaTeX/Source_Code_Listings

\definecolor{mygray}{rgb}{0.5,0.5,0.5}

\renewcommand{\lstlistingname}{Código-Fonte}

\lstdefinestyle{code_style}{
language=C,
breaklines=true,
frame=tb,
commentstyle=\color{mygray},
basicstyle=\footnotesize,
numbers=left,
numbersep=5pt,
numberstyle=\color{mygray},
tabsize=2,
showtabs=false
}
%%%%%%%%%%%%%%%%%%%%%%%%%%%%%%%%%%%%%%%%%%%%%

\lstinputlisting[caption=esqueleto\_codigo.h, style=code_style, label=lst:codigo_h]{Codigos/esqueleto_codigo.h}
\lstinputlisting[caption=esqueleto\_codigo.c, style=code_style, label=lst:codigo_c]{Codigos/esqueleto_codigo.c}
