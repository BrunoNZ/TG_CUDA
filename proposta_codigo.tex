\chapter{Proposta de um esqueleto de código padrão}

Por fim, resumindo-se tudo o que foi já foi dito e usando como base a implementação em CUDA, que se mostrou a mais eficiente, é apresentado um esqueleto padrão de código que soluciona o problema proposto de forma simplificada. Esse esqueleto visa facilitar o trabalho de adaptação de algoritmos sequenciais de processamento de dados atmosféricos em uma solução paralela do mesmo.

A implementação contém duas estruturas de variáveis, definidas no arquivo \ref{lst:estruturas}, nomeadas \texttt{parametros} e \texttt{parametros\_exec} que armazenam, respectivamente, os parâmetros de entrada e os parâmetros de execução da função em CUDA.

%%%%%%%%%%%%%%%%%%%%%%%%%%%%%%%%%%%%%%%%%%%%%%%%%%%%%%%%%%
% CODIGO:
\lstinputlisting[caption=Estrutura de Dados, style=code_style, label=lst:estruturas]{Codigos/esqueleto_estruturas.h}
%%%%%%%%%%%%%%%%%%%%%%%%%%%%%%%%%%%%%%%%%%%%%%%%%%%%%%%%%%

A função que será executada no device não difere muito da implementação sequencial, como já foi demonstrado na figura \ref{fig:comparacao_codigo_padrao}. Primeiramente é necessário adicionar a declaração \texttt{\_\_global\_\_} antes do nome da função, para que o compilador saiba que essa será a função a ser executada pelo device. Depois é preciso definir qual quadrícula cada thread processará, verificar se essa quadrícula está dentro do escopo do ciclo ou não e por fim calcular a posição do vetor do vetor que contém o início dos dados da quadrícula a ser processada. Essas três etapas estão definidas nas linhas 3, 4 e 6 do código-fonte \ref{lst:funcao}. Por fim há a implementação do algoritmo em si. Durante essa etapa só é necessário alguma alteração se o algoritmo possuir alguma função que não seja compatível com a arquitetura CUDA. Nesse caso é preciso encontrar uma função equivalente nas bibliotecas CUDA.

%%%%%%%%%%%%%%%%%%%%%%%%%%%%%%%%%%%%%%%%%%%%%%%%%%%%%%%%%%
% CODIGO:
\lstinputlisting[caption=Função que implementa o algoritmo, style=code_style, label=lst:funcao]{Codigos/esqueleto_funcao.c}
%%%%%%%%%%%%%%%%%%%%%%%%%%%%%%%%%%%%%%%%%%%%%%%%%%%%%%%%%%

Na função \texttt{main}, definida no código-fonte \ref{lst:main}, a primeira coisa a se fazer é ler os argumentos e os dados de entrada e calcular parâmetros de execução.

Os parâmetros de entrada são lidos pela função \texttt{le\_parametros\_entrada} e dados são lidos e reorganizados, da forma descrita no capitulo \ref{cap:proposta_solucao}, pela função \texttt{le\_matriz\_entrada} e são apontados pela variável \texttt{d\_entrada}. Já os parâmetros de execução são calculados pela função \texttt{calcula\_parametros\_execucao}, que tem como argumentos o número de pontos e o tamanho das séries do dado de entrada. Essas funções estão definidas respectivamente nas linhas 14, 23 e 46 do código-fonte \ref{lst:complementar}.

Após essas etapas é preciso inicializar e configurar o device para que o mesmo seja capaz de executar as tarefas necessárias. Esse processo começa com a alocação de espaço na memória no device para armazenar os dados de entrada e saída, igualmente como é feito em qualquer outro programa, e para isso é utilizado a função \texttt{cudaMalloc} da biblioteca CUDA, a qual se assemelha muito com a função \texttt{malloc}. A grande diferença desse passo para o que normalmente é feito é que o espaço alocado não é para armazenar todo o dado de entrada mas apenas uma parte desse referente ao que será processado em cada ciclo.

Logo após esse processo inicia-se o laço de ciclos. Em cada um desses ciclos serão executados os passos descritos na seção \ref{cap:implementacao_solucao}. Os passos estão definidos em ordem nas linhas 20, 22, 25, 30 e 32 do código-fonte \ref{lst:main}.

%%%%%%%%%%%%%%%%%%%%%%%%%%%%%%%%%%%%%%%%%%%%%%%%%%%%%%%%%%
% CODIGO:
\lstinputlisting[caption=Função \texttt{main}, style=code_style, label=lst:main]{Codigos/esqueleto_main.c}
%%%%%%%%%%%%%%%%%%%%%%%%%%%%%%%%%%%%%%%%%%%%%%%%%%%%%%%%%%

O cálculo da posição de memória, tanto para os dados de entrada quanto os saída, é feito multiplicando-se o ciclo atual, a quantidade de quadrículas a serem processadas em cada ciclo e o tamanho de cada série. A cópia os dados entre o host e o device é feito com a função \texttt{cudaMemcpy}.

Para executar o algoritmo em si é necessário passar como parâmetro, além dos parâmetros usuais, a quantidade de blocos por grid e de threads por bloco. Essas informações serão usadas pelo device para organizar as threads a serem executadas dentro do hardware disponível no device. O número de threads por bloco é definida no início do arquivo \ref{lst:complementar} como uma constante de nome \texttt{THREADS\_POR\_BLOCO}. A alteração desse valor pode trazer uma melhora de desempenho, no entanto depende diretamente do device em que o código será executado. Para facilitar o uso do código foi escolhido um valor padrão com o qual foi obtido em média os melhores resultados em todos os devices testados. O número de threads por bloco é calculado pela função \texttt{calcula\_parametros\_execucao}, como já descrito anteriormente. Por fim, a função \texttt{cudaDeviceSynchronize} garante que a execução do código principal só continue quando o todo o processamento no device termine para evitar que resultados ainda não calculados sejam copiados do device para o host.

Quando todos os ciclos terminarem, os resultados são escritos em disco pela função \linebreak \texttt{salva\_arq\_saida} no mesmo formato do dado de entrada original e os espaços em memória são liberados.

%%%%%%%%%%%%%%%%%%%%%%%%%%%%%%%%%%%%%%%%%%%%%%%%%%%%%%%%%%
% CODIGO:
\lstinputlisting[caption=Funções Complementares, style=code_style, label=lst:complementar]{Codigos/esqueleto_complementar.c}
%%%%%%%%%%%%%%%%%%%%%%%%%%%%%%%%%%%%%%%%%%%%%%%%%%%%%%%%%%

