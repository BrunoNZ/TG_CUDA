\chapter{Proposta de um esqueleto de código padrão}

Por fim, tendo a implementação em CUDA como base por ter se mostrado a melhor solução, 

O código é constituído por duas estruturas de variáveis, nomeadas $parametros$ e $parametros\_exec$ que armazenam, respectivamente, os parâmetros de entrada e os parâmetros de execução da função em CUDA. Por sua vez, os parâmetros de execução são calculados pela função $calcula\_parametros\_execucao$, que tem como argumentos o número de pontos e o tamanho das séries do dado de entrada.

Os dados de entrada são lidos, e reorganizados da forma previamente descrita, pela função $le\_matriz\_entrada$. Esses dados são apontados pelo variável $d\_entrada$.

%%%%%%%%%%%%%%%%%%%%%%%%%%%%%%%%%%%%%%%%%%%%%
%Definicao do estilo dos codigos
%Baseado em:
%http://en.wikibooks.org/wiki/LaTeX/Source_Code_Listings

\definecolor{mygray}{rgb}{0.5,0.5,0.5}

\lstdefinestyle{code_style}{
language=C,
breaklines=true,
frame=tb,
commentstyle=\color{mygray},
basicstyle=\footnotesize,
numbers=left,
numbersep=5pt,
numberstyle=\color{mygray},
tabsize=2,
showtabs=false
}
%%%%%%%%%%%%%%%%%%%%%%%%%%%%%%%%%%%%%%%%%%%%%

\lstinputlisting[caption=esqueleto\_codigo.h, style=code_style]{Codigos/esqueleto_codigo.h}
\lstinputlisting[caption=esqueleto\_codigo.c, style=code_style]{Codigos/esqueleto_codigo.c}